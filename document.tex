%%%%%%%%%%%%%%%%%%%%%%%%%%%%%%%%%%%%%%%%%%%%%%%%%%%%%%%%%%%%%%%%%%%%%%%%%%%%%%%%
% Arsclassica Article
% LaTeX Template
% Version 1.1 (10/6/14)
%
% This template has been downloaded from:
% http://www.LaTeXTemplates.com
%
% Original author:
% Lorenzo Pantieri (http://www.lorenzopantieri.net) with extensive modifications by:
% Vel (vel@latextemplates.com)
%
% License:
% CC BY-NC-SA 3.0 (http://creativecommons.org/licenses/by-nc-sa/3.0/)
%
%%%%%%%%%%%%%%%%%%%%%%%%%%%%%%%%%%%%%%%%%%%%%%%%%%%%%%%%%%%%%%%%%%%%%%%%%%%%%%%%

%-------------------------------------------------------------------------------
%	PACKAGES AND OTHER DOCUMENT CONFIGURATIONS
%-------------------------------------------------------------------------------

\documentclass[
10pt, % Main document font size
a4paper, % Paper type, use 'letterpaper' for US Letter paper
oneside, % One page layout (no page indentation)
%twoside, % Two page layout (page indentation for binding and different headers)
headinclude,footinclude, % Extra spacing for the header and footer
BCOR5mm, % Binding correction
]{scrartcl}

\input{structure.tex} % Include the structure.tex file which specified the document structure and layout

\hyphenation{Fortran hy-phen-ation} % Specify custom hyphenation points in words with dashes where you would like hyphenation to occur, or alternatively, don't put any dashes in a word to stop hyphenation altogether

%-------------------------------------------------------------------------------
%	TITLE AND AUTHOR(S)
%-------------------------------------------------------------------------------

\project{Documento de Requisitos}

\title{\normalfont\spacedallcaps{SISTEMA DE APOIO ÀS OLIMPÍADAS E PARAOLIMPÍADAS (SAPOP)}} % The article title
%\title{Sistema de Apoio às Olimpíadas e Paraolimpíadas (SAPOP)} % The article title

%\author{\spacedlowsmallcaps{John Smith* \& James Smith\textsuperscript{1}}} % The article author(s) - author affiliations need to be specified in the AUTHOR AFFILIATIONS block
\author{\spacedlowsmallcaps{Jean Amaro (8532401), Wesley Tiozzo (8077925) \string& Danilo Zecchin Nery (8602430)}}

\date{\today} % An optional date to appear under the author(s)

%-------------------------------------------------------------------------------

\begin{document}

%-------------------------------------------------------------------------------
%	HEADERS
%-------------------------------------------------------------------------------

\renewcommand{\sectionmark}[1]{\markright{\spacedlowsmallcaps{#1}}} % The header for all pages (oneside) or for even pages (twoside)
%\renewcommand{\subsectionmark}[1]{\markright{\thesubsection~#1}} % Uncomment when using the twoside option - this modifies the header on odd pages
\lehead{\mbox{\llap{\small\thepage\kern1em\color{halfgray} \vline}\color{halfgray}\hspace{0.5em}\rightmark\hfil}} % The header style

\pagestyle{scrheadings} % Enable the headers specified in this block

%-------------------------------------------------------------------------------
%	TABLE OF CONTENTS & LISTS OF FIGURES AND TABLES
%-------------------------------------------------------------------------------

\maketitle % Print the title/author/date block

\tableofcontents % Print the table of contents

%\setcounter{tocdepth}{2} % Set the depth of the table of contents to show sections and subsections only

%\listoffigures % Print the list of figures

%\listoftables % Print the list of tables

%-------------------------------------------------------------------------------
%	ABSTRACT
%-------------------------------------------------------------------------------

%\section*{Abstract} % This section will not appear in the table of contents due to the star (\section*)

%-------------------------------------------------------------------------------
%	AUTHOR AFFILIATIONS
%-------------------------------------------------------------------------------

%{\let\thefootnote\relax\footnotetext{* \textit{Department of Biology, University of Examples, London, United Kingdom}}}

%{\let\thefootnote\relax\footnotetext{\textsuperscript{1} \textit{Department of Chemistry, University of Examples, London, United Kingdom}}}

%-------------------------------------------------------------------------------

\newpage % Start the article content on the second page, remove this if you have a longer abstract that goes onto the second page

%-------------------------------------------------------------------------------
%	INTRODUCTION
%-------------------------------------------------------------------------------

\section{Introdução}\label{intro}

%A statement\footnote{Example of a footnote} requiring citation \cite{Figueredo:2009dg}.
%Some mathematics in the text: $\cos\pi=-1$ and $\alpha$.

\subsection{Propósito}
O propósito deste documento de especificação de requisitos é definir os requisitos do
sistema SAPOP - Sistema de Apoio às \textit{Olimpíadas} e \textit{Paraolimpíadas}, que tem como objetivos
principais auxiliar o acesso aos \textit{eventos} dos \textit{jogos} que ocorrerão nas \textit{Olimpíadas} e \textit{Paraolimpíadas} do Rio de Janeiro de 2016.

\subsection{Escopo}
O SAPOP permite o cadastramento e a consulta de \textit{eventos} dos \textit{jogos} olímpicos e paralímpicos,
\textit{comitês} olímpicos e paralímpicos, jogadores, \textit{patrocinadores}, \textit{apoiadores} e \textit{espectadores}.
A \textit{organização} das \textit{Olimpíadas} poderá gerenciar os \textit{eventos} (local e data), além de distribuir
os \textit{patrocinadores} e \textit{apoiadores} cadastrados entre os \textit{eventos}, fabricação de produtos licenciados
e aluguel de propriedades intelectuais (logo dos \textit{jogos}, mascote). Cada \textit{comitê} terá um cadastro
de seus jogadores e \textit{eventos} que eles participarão. Após o gerenciamento de cada \textit{evento}, será
aberto ao público uma interface de acesso para que haja a consulta e compra de ingresso dos \textit{eventos}.

\subsection{Definições, Acrônimos e Abreviações}
\begin{description}[noitemsep] % [noitemsep] removes whitespace between the items for a compact look
	\item[Olimpíadas] Conjunto de eventos e jogos que serão promovidos pela organização das Olimpíadas.
	\item[Paraolimpíadas] Conjunto de eventos e jogos que serão promovidos pela organização das Paraolimpíadas.
	\item[Patrocinador] Indivíduo, ou empresa, que arca com os custos da realização de um espetáculo, competição esportiva, programa de televisão ou de rádio etc., com objetivos de publicidade.
	\item[Apoiador] Indivíduo, ou empresa, que divulga a realização de um espetáculo, competição esportiva, programa de televisão ou de rádio etc., com objetivos de publicidade, porém, sem arcar com os custos.
	\item[Espectador] Aquele que assiste a um evento ou jogo das Olimpíadas ou Paraolimpíadas.
	\item[Comitê] Reunião de pessoas que visam o interesse das Olimpíadas ou Paraolimpíadas.
	\item[Jogo] Atividade, submetida a regras que estabelecem quem vence e quem perde.
	\item[Evento] Acontecimento (festa, espetáculo, comemoração, solenidade etc.) organizado pela organização, com objetivos comunitários ou promocionais.
	\item[Organização] Entidade que serve à realização de ações de interesse social, político etc.; instituição, órgão, organismo, sociedade. Para este sistema, a organização é formada pelos membros do COI e IPC (Comitês Internacionais Olímpicos e Paralímpicos).
	%\item[Licença] Permissão formal que uma autoridade constituída concede para o estabelecimento de indústria ou comércio, ou para o exercício de certas atividades.
	\item[Item de Informação] Cada informação armazenada em uma entidade que usufruirá do sistema. As informações mais comuns são: nome, sobrenome, CPF, telefone fixo, telefone celular, e-mail, senha, país, CEP, endereço e número, cidade, bairro, e estado.
	\item[Browser] Programa que permite a usuário da internet consultar páginas de hipertexto e navegar, passando de um ponto a outro da mesma página ou de página diferente, usando os links de hipertexto, além de desfrutar de outros recursos dessa rede de computadores; navegador.
\end{description}
 
\subsection{Organização da Especificação de Requisitos de Software}
Este documento está dividido em três seções. Na Seção \ref{intro}, uma breve introdução sobre
o conteúdo deste documento foi apresentada. Na Seção \ref{desc}, uma descrição geral do sistema
é apresentada. Na Seção \ref{reqs}, os requisitos específicos do SAPOP são descritos.

%-------------------------------------------------------------------------------
%	GENERAL DESCRIPTION
%-------------------------------------------------------------------------------

\section{Descrição Geral do SAPOP}\label{desc}
O Sistema de Apoio às \textit{Olimpíadas} e \textit{Paraolimpíadas} (SAPOP) tem como objetivo principal
auxiliar o gerenciamento dos \textit{jogos} olímpicos e e paralímpicos. Trata-se de um sistema
multiusuário, apesar de ser implementado em um ambiente Web. Os usuários deste sistema
são, principalmente, organizadores dos \textit{jogos}, \textit{comitês} olímpicos e paralímpicos, jogadores,
\textit{patrocinadores} e o público geral. Os organizadores dos \textit{jogos} poderão gerenciar os \textit{jogos},
definindo seus \textit{patrocinadores} e \textit{apoiadores}. Os \textit{comitês} olímpicos e paralímpicos poderão
gerenciar os jogadores e quais \textit{eventos} eles participarão. Os \textit{patrocinadores} e \textit{apoiadores}
terão acesso a uma área de leilão de espaço e tempo dos \textit{eventos}, nos quais poderão dar
lances sobre. O público terá acesso ao cronograma dos e ventos, e seus \textit{patrocinadores},
\textit{apoiadores}, jogadores, venda de ingressos e produtos licenciados.

\subsection{Funções do Produto}
O SAPOP tem por objetivo auxiliar os organizadores, através das seguintes funções:
\begin{itemize}[noitemsep]
	\item Gerenciar \textit{eventos} - adicionar, remover, consultar e atualizar \textit{eventos} e seus respectivos horários
	\item Gerenciar \textit{patrocinadores}/\textit{apoiadores} - adicionar, remover, consultar e atualizar \textit{patrocinadores} e \textit{apoiadores} que participarão do leilão de espaço e tempo dos \textit{eventos}
\end{itemize}
O SAPOP tem por objetivo auxiliar os \textit{comitês} olímpicos e paralímpicos, através das seguintes funções:
\begin{itemize}[noitemsep]
	\item Gerenciar os jogadores - adicionar, remover, consultar e atualizar jogadores que participarão dos \textit{eventos}
	\item Atribuir os jogadores aos \textit{eventos} criados pela \textit{organização}
\end{itemize}
O SAPOP tem por objetivo auxiliar os \textit{patrocinadores} e \textit{apoiadores}, através das seguintes funções:
\begin{itemize}[noitemsep]
	\item Consultar os \textit{eventos} criados pela \textit{organização}
	\item Dar lances no espaço e tempo dos \textit{eventos} criados pela \textit{organização}, para que possam colocar seu logo
\end{itemize}
O SAPOP tem por objetivo auxiliar o público geral, através das seguintes funções:
\begin{itemize}[noitemsep]
	\item Consultar os \textit{eventos}, \textit{patrocinadores}, \textit{apoiadores} e jogadores
	\item Comprar ingressos e produtos licenciados
\end{itemize}

\subsection{Características do Usuário}
O SAPOP é um sistema destinado a vários perfis de usuário. Seus usuários precisam ter conhecimento sobre:
\begin{itemize}[noitemsep]
	\item Noções sobre uso de computadores pessoais/mobile e plataformas Web
\end{itemize}

\subsection{Suposições e Dependências}
A configuração mínima requerida para a execução do SAPOP são micro computadores pessoais, memória RAM mínima de 1 GiB, ambiente Windows XP ou superior, ou Linux, ou MAC-OS e resolução mínima de 1280x800 pixels. Quanto às plataformas mobile, memória RAM mínima de 512 MiB, ambiente Windows Phone, ou Android, ou iOS e resolução mínima de 768x1366 pixeis. Taxa de transferência de rede mínima de 512 Kib/segundo.
O navegador da plataforma escolhida deve dar suporte à:
\begin{itemize}[noitemsep]
	\item HTML5
	\item CSS3
	\item JavaScript >= 1.8
\end{itemize}
 
%-------------------------------------------------------------------------------
%	METHODS
%-------------------------------------------------------------------------------

%\section{Methods}

%Reference to Figure~\vref{fig:gallery}. % The \vref command specifies the location of the reference

%\begin{figure}[tb]
%\centering 
%\includegraphics[width=0.5\columnwidth]{GalleriaStampe} 
%\caption[An example of a floating figure]{An example of a floating figure (a reproduction from the \emph{Gallery of prints}, M.~Escher,\index{Escher, M.~C.} from \url{http://www.mcescher.com/}).} % The text in the square bracket is the caption for the list of figures while the text in the curly brackets is the figure caption
%\label{fig:gallery} 
%\end{figure}

%\begin{enumerate}[noitemsep] % [noitemsep] removes whitespace between the items for a compact look
%\item First item in a list
%\item Second item in a list
%\item Third item in a list
%\end{enumerate}

%-------------------------------------------------------------------------------

%\subsection{Paragraphs}

%\lipsum[6] % Dummy text

%\paragraph{Paragraph Description} \lipsum[7] % Dummy text

%\paragraph{Different Paragraph Description} \lipsum[8] % Dummy text

%-------------------------------------------------------------------------------

%\subsection{Math}

%\lipsum[4] % Dummy text

%\begin{equation}
%\cos^3 \theta =\frac{1}{4}\cos\theta+\frac{3}{4}\cos 3\theta
%\label{eq:refname2}
%\end{equation}

%\lipsum[5] % Dummy text

%\begin{definition}[Gauss] 
%To a mathematician it is obvious that
%$\int_{-\infty}^{+\infty}
%e^{-x^2}\,dx=\sqrt{\pi}$. 
%\end{definition} 

%\begin{theorem}[Pythagoras]
%The square of the hypotenuse (the side opposite the right angle) is equal to the sum of the squares of the other two sides.
%\end{theorem}

%\begin{proof} 
%We have that $\log(1)^2 = 2\log(1)$.
%But we also have that $\log(-1)^2=\log(1)=0$.
%Then $2\log(-1)=0$, from which the proof.
%\end{proof}

%-------------------------------------------------------------------------------
%	SPECIFIC REQUIREMENTS
%-------------------------------------------------------------------------------

\section{Requisitos Específicos}\label{reqs}

\subsection{Requisitos Funcionais}
O SAPOP é um sistema independente que deve ter uma interface gráfica do usuário baseado
no ambiente Web. Além disso, é multiusuário e protegido por senha em áreas específicas
(compras de ingresso, gerenciamento de \textit{eventos} e notícias). Assim, o sistema não deve
permitir a realização de nenhuma funcionalidade por pessoas não autorizadas. Assim,
quanto à instalação do sistema, com excessão do público geral que optar por não
comprar ingressos, todos terão uma senha. Os Requisitos Funcionais do SAPOP estão
organizados com base nas principais funcionalidades do sistema: Área de Leilão,
Gerenciamento de \textit{Jogos}, Gerenciamento de \textit{Patrocinadores} e \textit{Apoiadores}, Consultas
Gerais, Loja de Produtos Oficiais e Ingressos.

\subsubsection{Requisitos para auxiliar os organizadores gerenciarem \textit{eventos}, \textit{patrocinadores} e \textit{apoiadores}}
\begin{enumerate}[noitemsep]
	\item O sistema deve permitir a inserção de \textit{eventos}, solicitando de seus organizadores os itens de informação necessários para a inserção de \textit{eventos}. Os itens de informação são: nome, local de competição, modalidade esportiva, tipo de provas, data, horário (início/término), e regras do \textit{jogo}.
	\item O sistema deve permitir a alteração e exclusão de \textit{eventos}.
	\item O sistema deve permitir a consulta de \textit{eventos}, solicitando de seus organizadores os itens de informação necessários para a consulta de \textit{eventos}. Os itens de informação são: nome, local de competição, modalidade esportiva ou data.
	\item O sistema deve permitir a inserção de \textit{patrocinadores} e \textit{apoiadores} que participarão do leilão de espaços e tempos dos \textit{eventos}, solicitando de seus organizadores os itens de informação necessários para a inserção de \textit{patrocinadores} e \textit{apoiadores}. Os itens de informação são: nome e tipo (\textit{patrocinador}/\textit{apoiador}).
	\item O sistema deve permitir a alteração e exclusão de \textit{patrocinadores} e \textit{apoiadores} que participarão do leilão de espaço e tempo dos \textit{eventos}.
	\item O sistema deve permitir a consulta de \textit{patrocinadores}, \textit{apoiadores}, solicitando de seus organizadores o \textit{item de informação} e seu tipo (\textit{patrocinadores}/\textit{apoiador}).
\end{enumerate}
\subsubsection{Requisitos para auxiliar os \textit{comitês} olímpicos e paralímpicos}
\begin{enumerate}[noitemsep]
	\item O sistema deve permitir a inserção de jogadores, solicitando de seus \textit{comitês} os itens de informação necessários para a inserção de jogadores. Os itens de informação são: nome completo, nacionalidade, data de nascimento, local de nascimento, altura, peso e modalidade esportiva.
	\item O sistema deve permitir a alteração e exclusão de jogadores que participarão dos \textit{eventos}.
	\item O sistema deve permitir a consulta de jogadores, solicitando de seus \textit{comitês} os itens de informação necessários para a consulta de jogadores. Os itens de informação são: nome, modalidade esportiva ou nacionalidade.
	\item O sistema deve permitir a atribuição dos jogadores aos \textit{eventos} criados pela \textit{organização}.
\end{enumerate}
\subsubsection{Requisitos para auxiliar os \textit{patrocinadores} e \textit{apoiadores}}
\begin{enumerate}[noitemsep]
	\item O sistema deve permitir a consulta dos \textit{eventos} criados pela \textit{organização}, solicitando os itens de informação necessários para a consulta de \textit{patrocinadores} e \textit{apoiadores}. Os itens de informação são: nome, local de competição, modalidade esportiva, data, horário (início/término).
	\item O sistema deve permitir a participação em leilão de forma que possa ser dado lances nos espaço e tempo dos \textit{eventos} criados pela \textit{organização}, para que possam colocar seu logo.
\end{enumerate}
\subsubsection{Requisitos para auxiliar o público geral}
\begin{enumerate}[noitemsep]
	\item O sistema deve permitir a consulta dos \textit{eventos} através de uma tabela.
	\item O sistema deve permitir a consulta dos \textit{patrocinadores} e \textit{apoiadores} através de uma lista.
	\item O sistema deve permitir a consulta dos jogadores através de uma lista.
	\item O sistema deve permitir a compra de produtos licenciados, podendo consultar o prazo de entrega solicitando o \textit{item de informação} CEP.
	\item O sistema deve solicitar ao público geral os itens de informação necessários para a compra de ingressos. Os itens de informação são: nome, sobrenome, CPF, telefone fixo, telefone celular, e-mail, senha, país, CEP, endereço e número, cidade, bairro, e estado. Dentre esses itens de informação, os obrigatórios são: nome, sobrenome, CPF, telefone fixo, telefone celular, e-mail, senha, país, CEP, endereço e número, cidade, bairro, e estado.
\end{enumerate}

\subsection{Requisitos de Desempenho}
\begin{enumerate}[noitemsep]
	\item O sistema deve apresentar tempo de resposta satisfatória para todas as funções requisitadas pelos organizadores, \textit{comitês}, \textit{patrocinadores}, \textit{apoiadores} e o público geral.
\end{enumerate}

\subsection{Atributos do Sistema de Software}
\subsubsection{Usabilidade}
\begin{enumerate}[noitemsep]
	\item O sistema deve fornecer uma interface amigável e acesso por meio da Web.
\end{enumerate}
\subsubsection{Confiabilidade}
\begin{enumerate}[noitemsep]
	\item O sistema deve encriptar mensagens enviadas ao sistema e dados armazenados pelo ele.
\end{enumerate}
\subsubsection{Eficiência}
\begin{enumerate}[noitemsep]
	\item O sistema deve apresentar ao usuário uma mensagem de sucesso ao fazer alguma alteração na base de dados, ou erro, caso aconteça.
\end{enumerate}
\subsubsection{Portabilidade}
\begin{enumerate}[noitemsep]
	\item O sistema deve ser executado nos \textit{browsers} descritos acima.
\end{enumerate}

%-------------------------------------------------------------------------------
%	RESULTS AND DISCUSSION
%-------------------------------------------------------------------------------

%\section{Results and Discussion}

%\lipsum[10] % Dummy text

%-------------------------------------------------------------------------------

%\subsection{Subsection}

%\lipsum[11] % Dummy text

%\subsubsection{Subsubsection}

%\lipsum[12] % Dummy text

%\begin{description}
%\item[Word] Definition
%\item[Concept] Explanation
%\item[Idea] Text
%\end{description}

%\lipsum[12] % Dummy text

%\begin{itemize}[noitemsep] % [noitemsep] removes whitespace between the items for a compact look
%\item First item in a list
%\item Second item in a list
%\item Third item in a list
%\end{itemize}

%\subsubsection{Table}

%\lipsum[13] % Dummy text

%\begin{table}[hbt]
%\caption{Table of Grades}
%\centering
%\begin{tabular}{llr}
%\toprule
%\multicolumn{2}{c}{Name} \\
%\cmidrule(r){1-2}
%First name & Last Name & Grade \\
%\midrule
%John & Doe & $7.5$ \\
%Richard & Miles & $2$ \\
%\bottomrule
%\end{tabular}
%\label{tab:label}
%\end{table}

%Reference to Table~\vref{tab:label}. % The \vref command specifies the location of the reference

%-------------------------------------------------------------------------------

%\subsection{Figure Composed of Subfigures}

%Reference the figure composed of multiple subfigures as Figure~\vref{fig:esempio}. Reference one of the subfigures as Figure~\vref{fig:ipsum}. % The \vref command specifies the location of the reference

%\begin{figure}[tb]
%\centering
%\subfloat[A city market.]{\includegraphics[width=.45\columnwidth]{Lorem}} \quad
%\subfloat[Forest landscape.]{\includegraphics[width=.45\columnwidth]{Ipsum}\label{fig:ipsum}} \\
%\subfloat[Mountain landscape.]{\includegraphics[width=.45\columnwidth]{Dolor}} \quad
%\subfloat[A tile decoration.]{\includegraphics[width=.45\columnwidth]{Sit}}
%\caption[A number of pictures.]{A number of pictures with no common theme.} % The text in the square bracket is the caption for the list of figures while the text in the curly brackets is the figure caption
%\label{fig:esempio}
%\end{figure}

%-------------------------------------------------------------------------------
%	BIBLIOGRAPHY
%-------------------------------------------------------------------------------

%\renewcommand{\refname}{\spacedlowsmallcaps{References}} % For modifying the bibliography heading

%\bibliographystyle{unsrt}

%\bibliography{sample.bib} % The file containing the bibliography

%-------------------------------------------------------------------------------

\end{document}
